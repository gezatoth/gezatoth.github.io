\documentclass[pra,amsmath,amssymb]{revtex4}

\usepackage{graphics}

\usepackage{epsfig}

\usepackage{bbm}

\newcommand{\bra}[1]{\ensuremath{\left\langle{#1}\right\vert}}
\newcommand{\braket}[1]{\ensuremath{\left\langle{#1}\right\rangle}}
\newcommand{\ket}[1]{\ensuremath{\left|{#1}\right\rangle}}
\newcommand{\ad}{\ensuremath{a^\dagger}}
\def\be{\begin{equation}}
\def\ee{\end{equation}}
\def\eea{\end{eqnarray}}
\def\bea{\begin{eqnarray}}
\newcommand{\bd}{\ensuremath{b^\dagger}}
\newcommand{\va}[1]{\ensuremath{(\Delta#1)^2}}
\newcommand{\varho}[1]{\ensuremath{(\Delta_\rho #1)^2}}
\newcommand{\ex}[1]{\ensuremath{\left\langle{#1}\right\rangle}}
\newcommand{\exrho}[1]{\ensuremath{\left\langle{#1}\right\rangle}_{\rho}}
\newcommand{\exs}[1]{\ensuremath{\langle{#1}\rangle}}
\newcommand{\eins}{\ensuremath{\mathbbm 1}}

\begin{document}
\title{Entanglement witnesses for detecting multi-qubit entanglement of many
  qubits\\(Motivations for quant-ph/0405165)}

%\author{G\'eza T\'oth (MPQ) and Otfried G\"uhne (Innsbruck)}

\maketitle

Quantum entanglement has been in the center of attention for
quantum physicist for more than half a century. Bell inequalities 
\cite{B64,M90,GB98}
provide a very insightful approach for detecting an important
characteristic arising from entanglement: non-locality. If a Bell
inequality is violated for an experiment, then it means that the
measurement results cannot be explained with a local hidden
variable theory.

Bell inequalities have two uses: (i) they demonstrate that quantum
mechanics as a theory is nonlocal. (ii) They prove that the
quantum state created in the experiment is entangled. Every state
violating a Bell inequality for some choice of observables  is
entangled. However, not all entangled states violate a Bell
inequality \cite{W89}. Thus there are entangled states which allow for a
local hidden variable model.

In the recent years powerful tools for detecting quantum
entanglement were developed: {\it entanglement witnesses} \cite{HH90}. These,
in principle, make possible detecting any entangled states. Unlike
Bell inequalities (which are classical), they use quantum
mechanics for obtaining conditions for entanglement. This is the
reason why in many situations {\it much fewer} measurements are
sufficient for entanglement detection than with Bell inequalities.

Let us provide a simple example. The well-known CHSH inequality
detects entangled states close to the state
$\Psi=(\ket{00}+\ket{11})/\sqrt{2}$. It requires measuring the
spin components $x$ and $y$ for both qubits. The CHSH inequality
claims that for states with local hidden variable model \be
\exs{x_1x_2}+\exs{y_1y_2}+\exs{x_1y_2}-\exs{y_1x_2}\le 2, \label{CHSH} \ee 
where $\exs{...}$ denotes expectation value.
The maximum
for local hidden variable models can be obtained by trying all the
$16$ possible combinations of $x_1,y_1,x_2,y_2=\pm 1$. The quantum
maximum of Eq. (\ref{CHSH}) is $2\sqrt{2}$.
Thus in an experiment the visibility must be
at least $2/(2\sqrt{2})\approx 70\%$

How can one see that Bell inequalities do not use quantum
mechanics? For example, they consider $\exs{x_k}=\exs{y_k}=+1$ a possible
measurement outcome. But from quantum mechanics we know that
$\exs{x_k}^2+\exs{y_k}^2\le 1$. That is, the maximal length of 
a Bloch vector is one.
If we use this knowledge then a much
simpler condition can be obtained. So for separable (not entangled)
states \be \exs{x_1x_2}+\exs{y_1y_2}\le 1.\label{ew}\ee 
This can be proved using that for product sates
$\exs{x_1x_2}+\exs{y_1y_2}=
\exs{x_1}\exs{x_2}+\exs{y_1}\exs{y_2}$.
The condition given in Eq. (\ref{ew}) is basically an
entanglement witness. If it is violated then the state is entangled.
Here the quantum maximum is $2$. Thus the
robustness of this condition is larger: it requires only
$50\%$ visibility in an experiment. Other advantage: only two
measurement settings are needed instead of four.

In the multi-qubit case the situation is the following \cite{GB98}. Let us
consider GHZ states for simplicity \cite{GH90}. 
The Mermin inequality \cite{M90} can be
used for entanglement detection around GHZ states. It also
requires detecting $x$ and $y$ for each qubit. The condition for
local hidden variable models for $n$ qubits is given as \bea
&&\exs{x_1x_2x_3x_4 \cdot \cdot \cdot
x_{N-1}x_{N}}\nonumber\\
&-&\exs{ y_{1}y_{2}x_{3}x_{4} \cdot \cdot \cdot x_{N-1}x_{N}}\nonumber\\
&+&\exs{y_{1}y_{2}y_{3}y_{4}\cdot \cdot \cdot x_{N-1}x_{N}}\nonumber\\
&-& ... \nonumber\\
&+&\exs{ y_{1}y_{2}y_{3}y_{4} \cdot \cdot \cdot
y_{N-1}y_{N}}\nonumber\\&\le& 2^{\lfloor n/2\rfloor},\eea where
$\lfloor x\rfloor$ denotes integer part. Here each term represents
the sum of all its possible permutations. Very many measurements
... What actually matters, is not the number of terms, but the
number of measurement settings. Here each term needs a separate
setting. That is, there are no two terms which could be measured
with the same setting.

%What does this inequality tells as if it is violated? It tells us
%that the system is entangled. But for a system of many spins
%(qubits) this is not enough information. Maybe only two qubit were
%entangled with each other and the rest stays in a product state.
%Because of that, conditions for detecting {\it genunie multi-party
%entanglement} were developed \cite{GB98}. 
%They require a higher bound for the
%violation of local realism. 

After the long introduction we reached the main point. It is
possible to design much more efficient 
conditions for entanglement using entanglement witnesses.
These detect multi-qubit entanglement with much much fewer
settings. In fact, they need only two settings. This is a
surprise. It is possible since the states most often considered in
quantum information (GHZ states, cluster states) are so-called
{\it stabilizer states}. Thus stabilizer theory, already very much used
in quantum error correction, can also be used for detecting
entanglement. For details please see Ref. \cite{TG04}.
\begin{thebibliography}{99}
% Bell inequalities
\bibitem{B64} J.S. Bell, Physics (Long Island City, N.Y.) {\bf 1}, 195 (1964).
\bibitem{M90} N.D. Mermin, Phys. Rev. Lett. {\bf 65}, 1838 (1990).
\bibitem{GB98}
N. Gisin, H. Bechmann-Pasquinucci, Phys. Lett A {\bf 246}, 1
(1998); D. Collins {\it et al.}, Phys. Rev. Lett. {\bf 88}, 170405
(2002); M. Seevinck and G. Svetlichny, 
Phys. Rev. Lett. {\bf 89}, 060401 (2002).
\bibitem{W89} R.F. Werner, Phys. Rev. A {\bf 40}, 4277 (1989).
% Witnesses
\bibitem{HH90} M. Horodecki {\it et al.},
Phys. Lett. A {\bf 223}, 1 (1996);
B. M. Terhal, Phys. Lett. A {\bf 271}, 319 (2000);
M. Lewenstein {\it et al.}, Phys. Rev. A {\bf 62}, 052310 (2000);
D. Bru{\ss} {\it et al.}, J. Mod. Opt. {\bf 49}, 1399 (2002);
 M. Bourennane {\it et al.},
Phys. Rev. Lett. {\bf 92}, 087902, (2004).
% GHZ state
\bibitem{GH90} D.M. Greenberger {\it et al.},
Am. J. Phys. {\bf 58}, 1131 (1990).
%\bibitem{MWITNESS} This bound is usually obtained
%from quantum mechanics. I.e., with the exception of
%the method presented in Ref. \cite{SG02},
%these inequalities are (not so
%efficient) entanglement witnesses.
% Short paper
\bibitem{TG04} G. T\'oth and O. G\"uhne, quant-ph/0405165;
quant-ph/0409132.
\end{thebibliography}
\end{document}
