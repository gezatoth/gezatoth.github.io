\documentclass[amsmath,amssymb]{revtex4}



\usepackage{graphics}



\usepackage{epsfig}



\usepackage{bbm}

\usepackage[latin1]{inputenc} % Umlaute!




%%%%%%%%%%%%%%%% General mathematical / physical commands %%%%%%%%%%%

\newcommand{\C}{\ensuremath{\mathbbm C}}
\newcommand{\be}{\begin{equation}}
\newcommand{\ee}{\end{equation}}
\newcommand{\eea}{\end{eqnarray}}
\newcommand{\bea}{\begin{eqnarray}}
\newcommand{\bd}{\ensuremath{b^\dagger}}
\newcommand{\va}[1]{\ensuremath{(\Delta#1)^2}}
\newcommand{\vasq}[1]{\ensuremath{[\Delta#1]^2}}
\newcommand{\varho}[1]{\ensuremath{(\Delta_\rho #1)^2}}
\newcommand{\ex}[1]{\ensuremath{\left\langle{#1}\right\rangle}}
\newcommand{\exs}[1]{\ensuremath{\langle{#1}\rangle}}
\newcommand{\exrho}[1]{\ensuremath{\left\langle{#1}\right\rangle}_{\rho}}
\newcommand{\eins}{\ensuremath{\mathbbm 1}}
\newcommand{\qed}{\ensuremath{\hfill \Box}}
\newcommand{\WW}{\ensuremath{\mathcal{W}}}
\newcommand{\PP}{\ensuremath{\mathcal{P}}}
\newcommand{\QQ}{\ensuremath{\mathcal{Q}}}
\newcommand{\KetBra}[1]{\ensuremath{| #1 \rangle \langle #1 |}}
\newcommand{\ketbra}[1]{\ensuremath{| #1 \rangle \langle #1 |}}
\newcommand{\Ket}[1]{\ensuremath{|#1\rangle}}
\newcommand{\ket}[1]{\ensuremath{|#1\rangle}}
\newcommand{\Bra}[1]{\ensuremath{\langle#1|}}
\newcommand{\bra}[1]{\ensuremath{\langle#1|}}
\newcommand{\BraKet}[2]{\ensuremath{\langle #1|#2\rangle}}
\newcommand{\braket}[2]{\ensuremath{\langle #1|#2\rangle}}
\newcommand{\KetBraO}[3]{\ensuremath{| #1 \rangle_{#3}\langle #2 |}}
\newcommand{\kommentar}[1]{}

\begin{document}
\title{Entanglement Witnesses using only Two Measurement Settings}
\date{\today}
\begin{abstract}
We will describe entanglement witnesses which
detect genuine multi-qubit entanglement close to GHZ and cluster states.
\end{abstract}

\author{G\'eza T\'oth}

\affiliation{Max Planck Institute for Quantum Optics,
E-mail: Geza.Toth@mpq.mpg.de}

\author{Otfried G\"uhne}

\affiliation{Theoretical Physics at Innsbruck,
E-mail: guehne@itp.uni-hannover.de}

\pacs{03.67.Mn, 03.65.Ud, 03.67.-a}

\maketitle


\section{Introduction}

These notes are written in order to present our entanglement
witnesses. Please look for more details in Ref. \cite{PAPER}. In
constructing the entanglement witnesses for cluster states and GHZ
states we use the stabilizing operators of these states. $S$ is a
stabilizing operator of $\ket{\Psi}$ if it satisfies \be S
\ket{\Psi} = \ket{\Psi}. \label{stabil} \ee

\section{Witnesses for the cluster state}

The stabilizing operators of an $N$-qubit cluster state
are \bea S_1^{(C_N)}&:=&\sigma_x^{(1)}
\sigma_z^{(2)},
\nonumber\\
S_k^{(C_N)}&:=&\sigma_z^{(k-1)} \sigma_x^{(k)} \sigma_z^{(k+1)};
k\in\{2,3,..,N-1\},
\nonumber\\
S_N^{(C_N)}&:=&\sigma_z^{(N-1)} \sigma_x^{(N)}. \label{eigenC}
\eea Our witness for the detection of $N$-qubit entanglement
around a cluster state is \bea \WW_{C_N} &:=& 3\cdot\eins- 2
\bigg[\prod_{\text{even k}}\frac{S_k^{(C_N)}+\eins}{2}+
\prod_{\text{odd k}}\frac{S_k^{(C_N)}+\eins}{2}\bigg]. \label{CN}
\eea If the expectation value of $\WW_{C_N}$ is negative then the
system was genuine $N$-qubit entangled. Two settings are needed.
These two settings are  the
$\{\sigma_x,\sigma_z,\sigma_x,\sigma_z,...\}$ setting and the
$\{\sigma_z,\sigma_x,\sigma_z,\sigma_x,...\}$ setting.

Our witness detects a state of the form \be \varrho(p)=p_{noise}
\cdot \eins / 2^N + (1-p_{noise}) \ketbra{C_N} \ee if \be
p_{noise}< \bigg\{
\begin{tabular}{ll}
$1/(4-4/2^{\frac{N}{2}})$ & if N is odd,\\
$1/[4-2(1/2^{\frac{N+1}{2}}+1/2^{\frac{N-1}{2}})]$ & if N is even.
\end{tabular}
\ee
At least $25 \%$ noise are tolerated. For lower $N$ the noise
tolerance is better as shown in Table I.

For $N=4$ noise is tolerated if $p_{noise}<0.33$. The witness is
\bea \WW_{C_4} &:=& 3\cdot\eins- \frac{1}{2}
\bigg(\sigma_x^{(1)}\sigma_z^{(2)}+\eins\bigg)
\bigg(\sigma_z^{(2)}\sigma_x^{(3)}\sigma_z^{(4)}+\eins\bigg) -
\frac{1}{2}
\bigg(\sigma_z^{(1)}\sigma_x^{(2)}\sigma_z^{(3)}+\eins\bigg)
\bigg(\sigma_z^{(3)}\sigma_x^{(4)}+\eins\bigg). \label{C4} \eea

For the $\ket{0000}+\ket{0011}+\ket{1100}-\ket{1111}$
state it is 
\bea \WW_{C_4} &:=& 3\cdot\eins- \frac{1}{2}
\bigg(\sigma_z^{(1)}\sigma_z^{(2)}+\eins\bigg)
\bigg(\sigma_z^{(2)}\sigma_x^{(3)}\sigma_x^{(4)}+\eins\bigg) -
\frac{1}{2}
\bigg(\sigma_x^{(1)}\sigma_x^{(2)}\sigma_z^{(3)}+\eins\bigg)
\bigg(\sigma_z^{(3)}\sigma_z^{(4)}+\eins\bigg). \label{CN2} \eea

Here $\sigma_x=[0,1; 1,0]$, 
$\sigma_y=[0,-i;i,0]$ and $\sigma_z=[1,0; 0,-1]$.
(I write this here since signs might be important.)

For this witness
\begin{equation}
\exs{\WW_{C_4}}\ge 2 (\frac{1}{2}-F).
\end{equation} 
hence a bound on fidelity $F:=Tr(\ketbra{C_4}\rho)$ can be obtained.

\begin{table}
\caption{Noise tolerance for the cluster state witness
vs. number of qubits}
\begin{tabular}{l|l}
N &  Noise tol.\\ \hline
2 &  0.50 \\
3 &  0.40 \\
4 &  0.33 \\
5 &  0.31 \\
6 &  0.29 \\
7 &  0.28 \\
8 &  0.27 \\
9 &  0.26 \\
10 & 0.26 \\
\end{tabular}
\end{table}

\section{A better witness for the cluster state}

The following witness tolerates $40\%$ noise for four qubits.
For the $\ket{0000}+\ket{0011}+\ket{1100}-\ket{1111}$
state it is 
\bea \WW_{C_4}' &:=& 4\cdot\eins- \frac{1}{2}
\bigg(\sigma_z^{(1)}\sigma_z^{(2)}+\eins\bigg)
\bigg(\sigma_z^{(2)}\sigma_x^{(3)}\sigma_x^{(4)}+\eins\bigg) -
\frac{1}{2}
\bigg(\sigma_x^{(1)}\sigma_x^{(2)}\sigma_z^{(3)}+\eins\bigg)
\bigg(\sigma_z^{(3)}\sigma_z^{(4)}+\eins\bigg)\nonumber\\&+&
\sigma_x^{(1)}\sigma_y^{(2)}\sigma_y^{(3)}\sigma_x^{(4)}+
\sigma_z^{(1)}\sigma_z^{(2)}\sigma_z^{(3)}\sigma_z^{(4)}. \label{CN2B} \eea
So now we have two extra terms and need four settings.
Notice that the constant changed from $3$ to $4$.
The new terms correspond to $-(S1*S4+S2*S3)$.

The stabilizing operators for our state are
\begin{eqnarray}
S_1&=&\sigma_z^{(1)}\sigma_z^{(2)},\nonumber\\
S_2&=&\sigma_x^{(1)}\sigma_x^{(2)}\sigma_z^{(3)},\nonumber\\
S_3&=&\sigma_z^{(2)}\sigma_x^{(3)}\sigma_x^{(4)},\nonumber\\
S_4&=&\sigma_z^{(3)}\sigma_z^{(4)}.
\end{eqnarray} 

For this witness
\begin{equation}
\exs{\WW_{C_4}'}\ge 4 (\frac{1}{2}-F).
\end{equation} 
hence a bound on fidelity can be obtained.
Notice the factor $4$ which was $2$ before for the other witness.
This new witness should give a better bound on the fidelity.

Another witness with the same noise tolerance.
\bea \WW_{C_4}'' &:=& 4\cdot\eins- \frac{1}{2}
\bigg(\sigma_z^{(1)}\sigma_z^{(2)}+\eins\bigg)
\bigg(\sigma_z^{(2)}\sigma_x^{(3)}\sigma_x^{(4)}+\eins\bigg) -
\frac{1}{2}
\bigg(\sigma_x^{(1)}\sigma_x^{(2)}\sigma_z^{(3)}+\eins\bigg)
\bigg(\sigma_z^{(3)}\sigma_z^{(4)}+\eins\bigg)\nonumber\\&-&
\sigma_y^{(1)}\sigma_y^{(2)}\sigma_z^{(4)}-
\sigma_z^{(1)}\sigma_y^{(3)}\sigma_y^{(4)}. \label{CN2C} \eea
Fidelity works the same way.
The new terms correspond to $-(S1*S3*S4+S1*S2*S4)$.

\section{Other ideas for 4qubit cluster}

Detection of any (i.e., not nesessarily genuine 4-qubit)
entanglement.
For separable states $\exs{S_k}+\exs{S_{k+1}}\le 1$ and 
$\exs{S_k}+\exs{S_{k+1}}+\exs{S_kS_{k+1}}\le 1$.
This is true for $k=1,2,3$. The first 
expression has a noise tolerance $1/2$, the second has $2/3$.
To check the first expression, you do not have to measure extra
things.

\section{Witnesses for the GHZ state}

The stabilizing operators of an $N$ qubit GHZ state are \bea
S_1^{(GHZ_N)}&:=& \prod_{k=1}^N \sigma_x^{(k)},
\nonumber\\
S_k^{(GHZ_N)}&:=&\sigma_z^{(k-1)} \sigma_z^{(k)}; \;\;
k\in\{2,3,..,N\}. \label{eigenGHZ} \eea With these operators our
witness looks like \be \WW_{GHZ_N} :=
3\cdot\eins-2\bigg[\frac{S_1^{(GHZ_N)}+\eins}{2}
+\prod_{k=2}^N\frac{S_k^{(GHZ_N)}+\eins}{2}\bigg].
\label{wnoisetol} \nonumber \ee If the expectation value of
$\WW_{GHZ_N}$ is negative then the system was genuine $N$-qubit
entangled.

The expression for $\WW_{GHZ_N}$ can be simplified using that \be
\prod_{k=2}^N\frac{S_k^{(GHZ_N)}+\eins}{2}= P_{\downarrow
\downarrow\downarrow ...}+ P_{\uparrow \uparrow\uparrow ...}, \ee
where $P_{\downarrow \downarrow\downarrow ...}$ is the projector
for  the state with all spins down and $P_{\uparrow
\uparrow\uparrow ...}$ is the projector for the state with all
spins up. Using these $\exs{\WW_{GHZ_N}}$ can then be measured as
follows. Only two (!) settings are needed. \be \exs{\WW_{GHZ_N}} =
2-\ex{\sigma_x^{(1)}\sigma_x^{(2)}\sigma_x^{(3)} \cdot \cdot \cdot
\sigma_x^{(N)}} -2\ex{ P_{\downarrow \downarrow\downarrow ...}+
P_{\uparrow \uparrow\uparrow ...}}. \label{wnoisetol2} \ee For the
first expectation value one has to measure the
$\{\sigma_x,\sigma_x,\sigma_x,...\}$ setting, for the second the
$\{\sigma_z,\sigma_z,\sigma_z,...\}$ setting.

It detects a state of the form \be \varrho(p)=p_{noise} \cdot
\eins / 2^N + (1-p_{noise}) \ketbra{GHZ_N} \ee if \be
p_{noise}<\frac{1}{3-4/2^N} \ee as true multipartite entangled.
Thus it tolerates at least $33\%$ noise. For lower $N$ the noise
tolerance is better as shown in Table II.

\begin{table}
\caption{Noise tolerance for the GHZ state witness vs. number of
qubits}
\begin{tabular}{l|l}
N &  Noise tol.\\ \hline
2 &  0.50 \\
3 &  0.40 \\
4 &  0.36 \\
5 &  0.35 \\
6 &  0.34 \\
7 &  0.34 \\
8 &  0.34 \\
9 &  0.33 \\
10 & 0.33 \\
\end{tabular}
\end{table}

\begin{thebibliography}{99}
\bibitem{PAPER}G\'eza T\'oth and Otfried G\"uhne,
Detecting Genuine Multipartite Entanglement
with Two Local Measurements, quant-ph/0405165.
\end{thebibliography}
\end{document}
